\documentclass[11pt]{amsart}
\usepackage{geometry}                % See geometry.pdf to learn the layout options. There are lots.
\geometry{letterpaper}                   % ... or a4paper or a5paper or ... 
%\geometry{landscape}                % Activate for for rotated page geometry
%\usepackage[parfill]{parskip}    % Activate to begin paragraphs with an empty line rather than an indent
\usepackage{graphicx}
\usepackage{amssymb}
\usepackage{epstopdf}
\DeclareGraphicsRule{.tif}{png}{.png}{`convert #1 `dirname #1`/`basename #1 .tif`.png}

\usepackage{amsmath}
\usepackage{amsthm}

\usepackage{stackengine}

\theoremstyle{definition}
\newtheorem{defn}{Definition}
\newtheorem*{defn*}{Definition}
\newtheorem*{conj*}{Conjecture}
\newtheorem{lemma}{Lemma}
\newtheorem*{cor*}{Corollary}
\newtheorem{cor}{Corollary}
\newtheorem*{lemma*}{Lemma}
\newtheorem*{note*}{Note}
\newtheorem{note}{Note}

\newcommand{\sgn}{\text{sgn}}

\title{Brief Article}
\author{The Author}
%\date{}                                           % Activate to display a given date or no date

\begin{document}
%\section{}
%\subsection{}



\begin{defn*}
Let $\mathcal{G}$ be a finite edge-weighted bipartite planar graph with a set {\bf N} of $2n$ special vertices called nodes on the outer face labeled $1, 2, \ldots, 2n$ in counterclockwise order. A double-dimer configuration on $(\mathcal{G}, {\bf N})$ is a configuration of disjoint loops, doubled edges, and simple paths that connect the nodes in pairs. Each configuration is weighted by the product of its edge-weights times $2^{\ell}$, where $\ell$ is the number of loops in the configuration.
\end{defn*}



\begin{defn*}
Let $\mathcal{G}^{BW}$ be the subgraph of $\mathcal{G}$ formed by deleting the nodes except for the ones that are black and odd or white and even. Let $\mathcal{G}^{BW}_{i, j}$ be defined as $\mathcal{G}^{BW}$ was, but with node $i$ included in $\mathcal{G}^{BW}_{i, j}$ if and only if it was not included in $\mathcal{G}^{BW}$, and similarly for node $j$. Let $Z^{BW}$ and $Z^{BW}_{i, j}$ be the weighted sum of dimer configurations of dimer configurations of $\mathcal{G}^{BW}$ and $\mathcal{G}^{BW}_{i, j}$, respectively. Define $Z^{WB}$ and $Z^{WB}_{i, j}$ similarly but with the roles of black and white reversed.
\end{defn*}


For each planar matching of the nodes $\sigma$, let $\text{Pr}(\sigma)$ be the probability that a random double-dimer configuration has pairing $\sigma$. Kenyon and Wilson showed that assuming all of the nodes are black and odd or white and even,  $\widehat{ \text{Pr} }(\sigma) := \text{Pr}(\sigma) \dfrac{ Z^{WB}}{Z^{BW}}$ is an integer coefficient  polynomial in the quantities $X_{i, j} := Z^{BW}_{i, j}/Z^{BW}$.


\begin{note*}
They assume that all odd nodes of $\mathcal{G}$ are black and all even nodes are white.
\end{note*}

\begin{defn*}
If $S$ is a balanced subset of nodes, that is, a subset containing an equal number $k$ of white and black nodes. Let $Z^{D}(S) = Z^{D}( \mathcal{G} \setminus S)$ be the weighted sum of dimer covers of $\mathcal{G} \setminus S$. Let $S^{c} = {\bf N} \setminus S$.
\end{defn*}

\begin{lemma*} [Lemma 3.1, KW 2006]
$Z^{D}(S) Z^{D}(S^{c})$ is a sum of double-dimer configurations for all connection topologies $\pi$ for which $\pi$ connects no element of $S$ to an element of $S^{c}$. That is,
$$Z^{D}(S) Z^{D}(S^{c}) = Z^{DD} \sum\limits_{\pi} M_{S, \pi} \Pr (\pi)$$
where $M_{S, \pi}$ is 0 or 1 according to whether $\pi$ connects nodes in $S$ to $S^{c}$ or not.
\end{lemma*}

Let $G$ be a graph with node set ${\bf N}$ that does not necessarily have the property that odd nodes of $\mathcal{G}$ are black and all even nodes are white. Let $M \subseteq N$ be the set of nodes that are odd and white or even and black.
Let $\tilde{G}$ be $G$ with an extra vertex and edge with weight 1 added to each node in $M$. So all of the nodes in $\tilde{G}$ are black and odd or white and even.

Additionally, assume that $G$ has an equal number of black and white nodes and $\tilde{G}$ has an equal number of black and white nodes. These two assumptions imply that $M$ is a balanced subset. Note that
$$Z^{D}(\tilde{G} \setminus S) = Z^{D}(G \setminus (S \cap M^{c} ) \cup (M \cap S^{c}) )$$

Then Lemma 3.1 says that

\begin{cor*}[Corollary to Lemma 3.1]
Let $S$ be a balanced subset of nodes. \\
$Z^{D}(G \setminus (S \cap M^{c} ) \cup (M \cap S^{c})) Z^{D}(G \setminus (S^{c} \cap M^{c} ) \cup (M \cap S))$ is a sum of double-dimer configurations for all connection topologies $\pi$ for which $\pi$ connects no element of $S$ to an element of $S^{c}$. That is,
$$Z^{D}(G \setminus (S \cap M^{c} ) \cup (M \cap S^{c})) Z^{D}(G \setminus (S^{c} \cap M^{c} ) \cup (M \cap S)) = Z^{DD}(G) \sum\limits_{\pi} M_{S, \pi} \Pr (\pi)$$
where $M_{S, \pi}$ is 0 or 1 according to whether $\pi$ connects nodes in $S$ to $S^{c}$ or not.
\end{cor*}

If $T = (S \cap M^{c}) \cup (M \cap S^{c})$, then $S = (T \cap M^{c}) \cup (M \cap T^{c})$, so:

\begin{cor*}[Corollary to Lemma 3.1]
Let $T$ be a balanced subset of nodes.
$Z^{D}(T) Z^{D}(T^{c})$ is a sum of all connection topologies $\pi$ for which $\pi$ connects no elements of $(T \cap M^{c}) \cup (M \cap T^{c})$ to $\left((T \cap M^{c}) \cup (M \cap T^{c})\right)^{c}$. That is
$$Z^{D}(G \setminus T) Z^{D}(G \setminus T^{c}) = Z^{DD}(G) \sum\limits_{\pi} M_{(T \cap M^{c}) \cup (M \cap T^{c}), \pi} \Pr (\pi)$$
where $M_{(T \cap M^{c}) \cup (M \cap T^{c}), \pi}$ is 0 or 1 depending on whether $\pi$ connects nodes in $(T \cap M^{c}) \cup (M \cap T^{c})$ to $\left((T \cap M^{c}) \cup (M \cap T^{c})\right)^{c}$.
\end{cor*}

Assuming that $\mathcal{G}$ has the property that the odd-numbered nodes are black and the even-numbered nodes are white, Kenyon and Wilson prove the following. 

\begin{lemma*}[Lemma 3.2, KW 2006]
Let $S$ be a balanced subset of $\{1, \ldots, 2n\}$. Then
$$\dfrac{Z^{D}(S) Z^{D}(S^{c})}{(Z^{D})^{2}} = 
\det[(1_{i, j \in S} + 1_{i, j \notin S}) \times (-1)^{(|i - j| -1)/2} X_{i, j} ]^{i = 1, 3, \ldots, 2n-1}_{j = 2, 4, \ldots, 2n}$$
\end{lemma*}

If we do not assume that $\mathcal{G}$ has the property that the odd-numbered nodes are black and the even-numbered nodes are white, we have the following conjecture:

\begin{conj*}
Let $\mathcal{G}$ be a bipartite planar graph with set of nodes $\{1, 2, \ldots, 2n\}$. Let $S$ be a balanced subset of $\{1, \ldots, 2n\}$. Then 
$$\dfrac{Z^{D}(\mathcal{G} \setminus S) Z^{D}(\mathcal{G} \setminus S^{c})}{(Z^{D}(\mathcal{G})^{2}} = 
\det \left[(1_{i, j \in S} + 1_{i, j \notin S}) \times \text{sign}(i, j, S) \dfrac{Z^{D}(\mathcal{G}_{i, j})}{Z^{D}(\mathcal{G})} \right]^{i = b_1, b_2, \ldots, b_n}_{j = w_1, w_2, \ldots, w_n}$$
where $b_1, b_2, \ldots, b_n$ are the black nodes of $\{1, 2, \ldots, 2n\}$ listed in ascending order,$w_1, w_2, \ldots, w_n$ are the white nodes of $\{1, 2, \ldots, 2n\}$ listed in ascending order, and $\text{sign}(i, j, S) = 1$ or $-1$ depending on $i, j,$ and $S$. 
\end{conj*}

\begin{proof}
Let $K$ be a Kasteleyn matrix of $\mathcal{G}$. Then the signs of the edges $\mathcal{G}$ are so that there are an odd number of $-$ signs if the face has 0 mod 4 edges and an even number of $-$ signs if the face has 2 mod 4 edges. 
Let $(w_1, b_1), \ldots, (w_k, b_k)$ be any noncrossing pairing of the nodes of $S$, where $w_1, \ldots, w_k$ are the white nodes of $S$ and $b_1, \ldots b_k$ are the black nodes of $S$ (such a noncrossing pairing always exists since $S$ is a balanced subset). 
Adjoin edges $e_i$ of weight $W$ to the outer face connecting $w_i$ to $b_i$ for $1 \leq i \leq k$. (What if two nodes are adjacent, and doing this creates a multigraph?). We claim that we can choose a weight for each $e_i$ that retains the Kasteleyn sign condition. 

To see this, first observe that adjoining the edge $e_1$ creates exactly two new faces, since $w_1$ and $b_1$ are on the outer face (i.e. the face bounding the infinite region) of $\mathcal{G}$. Let $p$ and $q$ be the two paths from $w_1$ to $b_1$ consisting of edges along the outer face of $\mathcal{G}$. Observe that $p$ and $q$ must both have odd length, since $w_1$ and $b_1$ are different colors. There are two different cases, depending on whether the outer face of $\mathcal{G}$ has 0 mod 4 edges or 2 mod 4 edges. 
\begin{enumerate}
\item The outer face of $\mathcal{G}$ has 0 mod 4 edges. \\
When the outer face of $\mathcal{G}$ has 0 mod 4 edges, one of the paths $p$ or $q$ must have length $4 \ell + 1$ for some $\ell$ and the other path must have length $4 j + 3$ for some $j$. Assume without loss of generality that $p$ has length $4 \ell + 1$ for some $\ell$. Since the outer face of $\mathcal{G}$ has 0 mod 4 edges, the product of the signs of the edges of the outer face of $\mathcal{G}$ is $-1$. This means that either the product of the signs of the edges of $p$ is 1 and the product of the signs of the edges of $q$ is $-1$, or the product of the signs of the edges of $p$ is $-1$ and the product of the signs of the edges of $q$ is $1$. So there are two cases: 
\begin{enumerate}
\item The product of the signs of the edges of $p$ is $1$. \\
Since the face bounded by the path $p$ and the edge $e_1$ has 2 mod 4 edges, give $e_1$ sign $1$. Then since the product of the signs of the edges of $q$ is $-1$, the face bounded by the path $q$ and the edge $e_1$ has 0 mod 4 edges and an odd number of $-$ signs.
\item The product of the signs of the edges of $p$ is $-1$.\\
Since the face bounded by the path $p$ and the edge $e_1$ has 2 mod 4 edges, give $e_1$ sign $-1$. Then since the product of the signs of the edges of $q$ is 1, the face bounded by the path $q$ and the edge $e_1$ has 0 mod 4 edges and an odd number of $-$ signs. 
\end{enumerate}
\item The outer face of $\mathcal{G}$ has 2 mod 4 edges. 
When the outer face of $\mathcal{G}$ has 2 mod 4 edges, either both $p$ and $q$ have length 1 mod 4 or both $p$ and $q$ have length 3 mod 4. 
Since the outer face of $\mathcal{G}$ has 2 mod 4 edges, the outer face has an even number of $-$ signs. This means that either both $p$ and $q$ have an even number of $-$ signs or both $p$ and $q$ have an odd number of minus signs. So there are several cases: 
\begin{enumerate}
\item $p$ and $q$ have length 1 mod 4
In this case, both the face bounded by the path $p$ and the edge $e_1$ and the face bounded by the path $q$ and the edge $e_1$ have 2 mod 4 edges. 
\begin{enumerate}
\item $p$ and $q$ both have an even number of $-$ signs
Give $e_1$ sign $1$. Then both new faces have an even number of $-$ signs. 
\item $p$ and $q$ both have an odd number of $-$ signs
Give $e_1$ sign $-1$. Then both new faces have an even number of $-$ signs. 
\end{enumerate}
\item $p$ and $q$ have length 3 mod 4
In this case, both the face bounded by the path $p$ and the edge $e_1$ and the face bounded by the path $q$ and the edge $e_1$ have 0 mod 4 edges. 
\begin{enumerate}
\item $p$ and $q$ both have an even number of $-$ signs
Give $e_1$ sign $-1$. Then both new faces have an odd number of $-$ signs. 
\item $p$ and $q$ both have an odd number of $-$ signs
Give $e_1$ sign $1$. Then both new faces have an odd number of $-$ signs. 
\end{enumerate}
\end{enumerate}
\end{enumerate}
So in all cases we have chosen the sign for $e_1$ so that the resulting graph still satisfies the Kasteleyn sign condition. By induction, we can add choose a sign for each $e_i$ so that the resulting graph satisfies the Kasteleyn sign condition. Let $K_{W}$ be the matrix defined as follows: $(K_{W})_{s, t} = \text{sign}(e_i) W$ if $(s, t) = (w_{i}, b_{i})$ for some $i$, and $(K_{W})_{s, t} = (K)_{s, t}$ otherwise. By construction, if $K_{0} := K_{W} |_{W = 0}$, then $K_0 = K$. 
 Reorder the columns of $K_{W}$ so that $w_1, \ldots, w_k$ are the first $k$ rows and $b_1, \ldots, b_k$ are the first $k$ columns. Since $K$ is a Kasteleyn matrix of $G$ and we chose signs of $e_i$ to retain the Kasteleyn sign condition, $Z^{D}(\mathcal{G} \setminus S) = \pm [W^{k}] \det(K_{W})$. 
We see that
$$\dfrac{ Z^{D}(\mathcal{G} \setminus S)}{Z^{D}(\mathcal{G})} = \dfrac{ [W^{k}] \det(K_{W}) }{[W^{0}] \det K_{W}}$$
Then
$$ \dfrac{ [W^{k}] \det(K_{W}) }{[W^{0}] \det K_{W}} = \pm \dfrac{ \det K_{\setminus S} }{ \det K }$$
how to determine sign here?
Not sure why any of this is necessary if I can't keep track of the sign. 
is $Z^{D}(\mathcal{G} \setminus S) = \pm \det (K_{\setminus S})$? Maybe not because although both $\mathcal{G}$ and $\mathcal{G} \setminus S$ have Kasteleyn matrices, if we let $K$ be the Kasteleyn matrix for $\mathcal{G}$, its not necessarily the case that the Kasteleyn matrix for $\mathcal{G} \setminus S$ is $K_{\setminus S}$. 

\end{proof}


\begin{conj*}
Let $\mathcal{G}$ be a bipartite planar graph with set of nodes $\{1, 2, \ldots, 2n\}$. Assume that $G$ has the property that the nodes alternate between black and white except for in two places, i.e. there are exactly two pairs of consecutive nodes that are both the same color: $(s, s+1)$, $(u,u+1)$. Assume without loss of generality that $s$ and $s+1$ are both black and $u$ and $u+1$ are both white. Let $S$ be a balanced subset of $\{1, \ldots, 2n\}$. Then 
$$\dfrac{Z^{D}(\mathcal{G} \setminus S) Z^{D}(\mathcal{G} \setminus S^{c})}{(Z^{D}(\mathcal{G})^{2}} = \pm
\det \left[(1_{i, j \in S} + 1_{i, j \notin S}) \times \text{sign}(i, j) \dfrac{Z^{D}(\mathcal{G}_{i, j})}{Z^{D}(\mathcal{G})} \right]^{i = b_1, b_2, \ldots, b_n}_{j = w_1, w_2, \ldots, w_n}$$
where $b_1, b_2, \ldots, b_n$ are the black nodes of $\{1, 2, \ldots, 2n\}$ listed in ascending order,$w_1, w_2, \ldots, w_n$ are the white nodes of $\{1, 2, \ldots, 2n\}$ listed in ascending order, and $\text{sign}(i, j) = (-1)^{(|b-w| -1)/2}$ if $b$ and $w$ are different parity and $\text{sign}(i, j) = (-1)^{|b-w|/2}$ if $b$ and $w$ are the same parity. If $i < s$ and $j > u$ or $j < s$ and $i < u$, $\text{sign}(i, j) = (-1)^{(|b-w| +1)/2}$ if $b$ and $w$ are different parity.
\end{conj*}


\begin{proof}
Without loss of generality (by rotating if necessary) we assume that $s = 1$.
Following the proof of Lemma 3.2 in Kenyon and Wilson, for convenience we adjoin to the graph $\mathcal{G}$ $2n-2$ edges as follows: we
connect nodes 2 and 3, 3 and 4, 4 and 5, $\ldots$, $u-1$ and $u$ with an edge, and we connect $u+1$ and $u+2$, $u+2$ and $u+3$, $\ldots$, $2n$ and $1$ with an edge. The resulting graph is still bipartite by the assumption that the nodes alternate between black and white except for the nodes $1$ and $2$ and the nodes $u$ and $u+1$. Now add four more edges as follows. Since $G$ is bipartite, there is a white vertex $t$ on the outer face of $G$ between nodes $1$ and $2$ and a black vertex $v$ on the outer face of $G$ between nodes $u$ and $u+1$. Add edges connecting nodes $1$ and $t$ and $t$ and $2$, and edges connecting nodes $u$ and $v$ and $v$ and $u+1$. Give the $2n+2$ edges we have added weight 0 (or weight $\epsilon$ and then take the limit $\epsilon \to 0$.) Given a Kasteleyn matrix of a graph, the signs of edges incident to a vertex may be reversed, and each face will still have a correct number of minus signs. For each $i = 2, 3, \ldots, u-1$, in order, if the edge from node $i$ to node $i+1$ has a minus sign, reverse the signs of all edges incident to node $i+1$. Do the same for the edge from node $u$ to $v$, the edge from $v$ to $u+1$, and the edges from node $i$ to node $i +1$ for $i = u+1, \ldots, 2n-1$. Do the same for the edge from node $2n$ to $1$ and the edge from $1$ to $t$. Then the sign of the remaining edge from $t$ to $1$ will be $(-1)^n$ (?) for the outer face to have the correct number of $-$ signs. 

Let $(w_1, b_1), \ldots, (w_k, b_k)$ be any noncrossing pairing of the nodes of $S$, where $w_1, \ldots, w_k$ are the white nodes of $S$ and $b_1, \ldots, b_k$ are the black nodes of $S$. Adjoin edges of weight $W$ connecting $w_i$ to $b_i$ for $1 \leq i \leq k$. To retain the Kasteleyn sign condition, the sign of a new edge connecting black node $b$ and white node $w$ should be $(-1)^{(|b-w|-1)/2}$ if $b$ and $w$ have different parity, and $(-1)^{(|b-w|)/2}$ if $b$ and $w$ have the same parity. For if nodes $b$ and $w$ have different parity, we must have $2 \leq b, w \leq u$ or $u+1 \leq w \leq 2n$ and $u+1 \leq b \leq 2n$ unless $b = 1$. Then because of the edges added in the previous paragraph, the sign of the new edge connecting $b$ and $w$ should be $(-1)^{(|b-w|-1)/2}$. If nodes $b$ and $w$ have the same parity, then this means that either $2 \leq w \leq u$ and $u+1 \leq b \leq 2n$ or $u = 1$, or $2 \leq b \leq u$ and $u+1 \leq w \leq 2n$. In this case, because of the edges between $u$ and $v$ and $v$ and $u+1$, the sign of the new edge connecting $b$ and $w$ should be $(-1)^{(|b-w|)/2}$. 

Rest of the proof of Lemma 3.2 should go through until the part about the global sign. 
\end{proof}

Recall that the cross of a pairing is a set of two parts ${a, c}$ and ${b, d}$ of $\rho$ such that $a < b < c < d$. 
Kenyon and Wilson define the parity of an odd-even pairing $\rho = ((1, w_1), (3, w_2), \ldots, (2n-1, w_n))$ to be the parity of the permutation $(w_1/2)(w_2/2)\ldots(w_n/n)$. Then they prove the following

\begin{lemma*}[Lemma 3.4, KW 2006]
For odd-even pairings $\rho$, 
$$(-1)^{\text{parity of } \rho} \prod\limits_{(i, j) \in \rho} (-1)^{(|i-j|-1)/2} = (-1)^{\# \text{ crosses of } \rho}$$
\end{lemma*}

Any pairing on $2n$ elements $\rho$ can be written $\rho = ((i_1, j_1), (i_2, j_2), \ldots, (i_n, j_n))$ so that $i_k < j_k$ for all $k$ and $i_1 < i_2 < \cdots < i_n$. We define $\sgn(\rho)$ to be the parity of the permutation $j_1 j_2 \ldots j_n$, when $j_1, \ldots j_n$ are relabeled with the labels $1, \ldots, n$ so that $\min_{1 \leq k \leq n} \{j_{k} \}$ is relabeled 1, $\ldots$, $\max_{1 \leq k \leq n} \{j_{k} \}$ is relabeled $n$.

\begin{conj*}
For any pairing $\rho$, 
$$\sgn(\rho) \prod\limits_{(i, j) \in \rho} f(i, j) =  (-1)^{\# \text{ crosses of } \rho}$$
where $f(i, j)$ is defined as follows: 
$$f(i, j) = \begin{cases} 
(-1)^{(|i-j| -1)/2} & \mbox{ if $i, j$ are different parity} \\
(-1)^{|i-j|/2} & \mbox{if $i, j$ are both odd} \\
(-1)^{(|i-j|-2)/2} & \mbox{if $i, j$ are both even} \\
\end{cases}$$
\end{conj*}

\begin{note*}
This conjecture has been checked for all pairings on 4, 6, 8, 10, 12, and 14 elements. 
\end{note*}

Observe that if $\sigma$ is an odd-even pairing on $2n$ elements, then $\sigma$ can be viewed as a map from the odd integers $1, \ldots, 2n-1$ to the even integers. A pairing $\tau$ can be viewed as a map from the even integers to the integers. If we relabel $\tau(\sigma(1)) \tau(\sigma(3)) \cdots \tau(\sigma(2n-1))$ so that the smallest number is labeled with $1$, the second smallest with 2, and so on, then we have a permutation on $n$ integers that we can find the sign of. 

\begin{conj*}
When a graph G with six nodes is colored as follows: Nodes 1, 4, and 5 are black and nodes 2, 3, and 6 are white, then
\begin{itemize}
\item $\footnotesize{\stackanchor{1}{2} } \big{|} \footnotesize{\stackanchor{4}{6} } \big{|} \footnotesize{\stackanchor{5}{3} } \equiv
\footnotesize{\stackanchor{1}{2} } \big{|} \footnotesize{\stackanchor{3}{4} } \big{|} \footnotesize{\stackanchor{5}{6} }  -  
\footnotesize{\stackanchor{1}{2} } \big{|} \footnotesize{\stackanchor{3}{6} } \big{|} \footnotesize{\stackanchor{5}{4} } $
\item $13| 24 | 56 \equiv 12|34|56 - 14|32|56$
\item $13 | 25 | 46 \equiv 12| 34 | 56 - 12| 36 | 54 - 14| 32 | 56 + 16 | 32 | 54$
\item $16 | 24 | 35 \equiv - 16| 32 | 54 + 16 | 34 | 52$
\end{itemize}
\end{conj*}

\begin{note*}
Idea for proof of first bullet: for any odd-even pairing $\rho$, $\langle \rho, LHS \rangle_{2} = 2^2$ or $2$, depending on whether 1 and 2 are paired in the odd-even pairing or not. If 1 and 2 are paired in the odd-even pairing, $\langle \rho, RHS \rangle_{2} = 2^3 - 2^2 = 2^2$ or $\langle \rho, RHS \rangle_{2} = -2^2$. If $1$ and $2$ are not paired in the odd-even pairing, $\langle \rho, RHS \rangle_{2} = 2$ or $-2$. Need to figure out signs. 
\end{note*}





\end{document}  